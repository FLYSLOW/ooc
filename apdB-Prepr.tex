% # Copyright (C) 2009-2011, suxpert, All right reserved.
% # -*- coding: utf-8 -*-
% !TEX encoding = UTF-8 Unicode

% Version Control System Information: Subversion, host on Google Code;
% FileID:		$Id$;
% FileDate:		$Date$;
% FileRevision:	$Revision$

% \chapter{The \cemph{ooc} Preprocessor ---Hints on \cemph{awk} Programming}
\chapter{\cemph{ooc} 预处理器:\cemph{awk} 编程指南}
\label{apd:Preprocessor}
\cemph{awk} was originally delivered with the Seventh Edition of the
\prop{UNIX} system. Around 1985 the authors extended the language
significantly and described the result in \cite{AWK88}. Today, there is a
\prop{POSIX} standard emerging and the new language is available in various
implementations, e.g., as \cemph{nawk} on \prop{System V}; as \cemph{awk},
adapted from the same sources, with the \prop{MKS}-Tools for \prop{MSDOS};
and as \cemph{gawk} from the (new) \cemph{awk} programming language and
provides an overview of the implementation of the \cemph{ooc} preprocessor.
The implementation uses several features of the \prop{POSIX} standard, and
it has been developed with \cemph{gawk}.

% \section{Architecture}
\section{构型}% Architecture
\cemph{ooc} is implemented as a shell script to load and execute an
\cemph{awk} program. The shell script facilitates passing \cemph{ooc}
command arguments to the \cemph{awk} program and it permits storing the
various modules in a central place.

The \cemph{awk} program collects a database of information about classes and
methods from the class description files, and produces C code from the
database for interface and representation files and for method headers,
selectors, parameter import, and initialization in the implementation files.
The \cemph{awk} program is based on two design concepts: modularisation and
report generation.

A module contains a number of functions and a \ccbf{BEGIN} clause defining
the global data maintained by the functions. \cemph{awk} does not support
information hiding, but the modules are kept in separate files to simplify
maintenance. The \cemph{ooc} command script can use \ccbf{AWKPATH} to locate
the files in a central place.

All work is done under control of \ccbf{BEGIN} clauses which \cemph{awk}
will execute in order of appearance. Consequently, \cemph{main.awk} must be
loaded last, because it processes the \cemph{ooc} command line.

Pattern clauses are not used. They cannot be used for all files anyway,
because \cemph{ooc} consults for each class description all class
description files leading up to it. The algorithm to read lines, remove
comments, and glue continuation lines together is implemented in a single
function \ccbf{get()} in \cemph{io.awk}. If pattern clauses were used, the
same algorithm would have to be replicated in pattern clauses.

The database can be inspected if certain debugging modules are loaded as
part of the \cemph{awk} program. These debugging modules use pattern clauses
for control, i.e., debugging starts once the command line processing of
\cemph{ooc} has been completed. Debugging statements are entered from
standard input and they are executed by the pattern clauses.

Regular output is produced only by interpreting reports. The design goal is
that the \cemph{awk} program contain as little information about the
generated code as possible. Code generation should be controlled totally by
means of changing the report files. Since the \cemph{ooc} command script
permits substitution of report files, the application programmer may modify
all output, at least theoretically, without having to change the \cemph{awk}
program.

% \section{File Management ---\cemph{io.awk}}
\section{文件管理:\cemph{io.awk}}
<++>

% \section{Recognition ---\cemph{parse.awk}}
\section{识别:\cemph{parse.awk}}
<++>

% \section{The Database}
\section{数据库}
<++>

% \section{Report Generation ---\cemph{report.awk}}
\section{报告生成:\cemph{report.awk}}
<++>

% \section{Line Numbering}
\section{行计数}
<++>

% \section{The Main Program ---\cemph{main.awk}}
\section{主程序:\cemph{main.awk}}
<++>

% \section{Report Files}
\section{报告文件}
<++>

A quick brown fox jumps over the lazy dogs.
A quick brown fox jumps over the lazy dogs.
A quick brown fox jumps over the lazy dogs.

A quick brown fox jumps over the lazy dogs.
A quick brown fox jumps over the lazy dogs.
A quick brown fox jumps over the lazy dogs.

% \section{The \cemph{ooc} Command}
\section{\cemph{ooc} 命令}
\cemph{ooc} can load an arbitrary number of reports and descriptions, output
several interface and representation files, and suggest or preprocess
various implementation files, all in one run. This is a consequence of the
modular implementation. However, \cemph{ooc} is a genuine filter, i.e., it
will read files as directed by the command line, but it will only write to
standard output. If several outputs are produced in one run, they would have
to be split and written to different files by a postprocessor based on
\cemph{awk} or \cemph{csplit}. Here are some typical invocations of
\cemph{ooc}:

\begin{lstlisting}[language=sh,morekeywords={ooc}]
$ ooc -R Object -h > Object.h		# root class
$ ooc -R Object -r > Object.r
$ ooc -R Object Object.dc > Object.c

$ ooc Point -h > Point.h			# other class
$ ooc -M Point Circle >> makefile	# dependencies
$ echo "Point.c: Point.d" >> makefile
$ ooc Circle -dc > Circle.dc		# start an implementation
$ ooc Circle -dc | ooc Circle - > Circle.c	# fake...
\end{lstlisting}
If \cemph{ooc} is called without arguments, it produces the following usage
description:
\begin{lstlisting}[language=,keywords={ooc}]
$ ooc
usage:	ooc [option ...] [report ...] description target ...
options:	 -d			arrange for debugging
			 -l			make #line stamps
			 -Dnm=val	define val for `nm (one word)
			 -M			make dependency for each description
			 -R			process root description
			 -7 -8 ...	versions for book chapters
report:		 report.rep	load alternative report file
description: class		load class description file
targets:	 -dc		make thunks for last 'class'
			 -h			make interface for last 'class'
			 -r			make representation for last 'class'
			 -			preprocess stdin for last 'class'
			 source.dc preprocess source for last 'class'
\end{lstlisting}
It should be noted that if any report file is loaded, the standard reports
are not loaded. The way to replace only a single standard report file is to
provide a file by the same name earlier on \ccbf{OOCPATH}.

The \cemph{ooc} command script needs to be reviewed during installation. It
contains \ccbf{AWKPATH}, the path for the \cemph{awk} processor to locate
the modules, and \ccbf{OOCPATH} to locate the reports. This last variable is
set to look in a standard place as a last resort; if \cemph{ooc} is called
with \ccbf{OOCPATH} already defined, this value is prefixed to the standard
place.

To speed things up, the command script checks the entire command line and
loads only the necessary report files. If \cemph{ooc} is not used correctly,
the script emits the usage description shown above. Otherwise \cemph{awk} is
executed by the same process.

% vim: set syntax=tex ts=4 sw=4 tw=76 fo+=Mm cc=+2 :

