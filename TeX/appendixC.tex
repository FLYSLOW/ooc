% # Copyright (C) 2009-2011, suxpert, All right reserved.
% # -*- coding: utf-8 -*-
% !TEX encoding = UTF-8 Unicode

% Version Control System Information: Subversion, host on Google Code;
% FileID:		$Id$;
% FileDate:		$Date$;
% FileRevision:	$Revision$

% \chapter{Manual}
\chapter{手册}
\label{apd:Manual}
This appendix contains \prop{UNIX} manual pages describing the final version
of \cemph{ooc} and some classes developed in this book.

% \section{Commands}
\section{命令}

% \subsection{munch ---produce class list}
\subsection{munch:生成类列表}
\begin{lstlisting}[keywords={nm,munch}]
	nm -p object... archive... | munch
\end{lstlisting}
\cemph{munch} reads a Berkeley-style \cemph{nm(1)} listing from standard
input and produces as standard output a C source file defining a
null-terminated array \ccbf{classes[]} with pointers to the class functions
found in each \cemph{object} and \cemph{archive}. The array is sorted by
class function names.

A class function is any name that appears with type \ccbf{T} and, preceded
with an underscore, with type \ccbf{b}, \ccbf{d}, or \ccbf{s}.

This is a hack to simplify retrieval programs. The compatible effect of
option \ccbf{-p} in Berkeley and System V \cemph{nm} is quite a surprise.

Because \prop{HP/UX} \cemph{nm} does not output static symbols,
\cemph{munch} is not very useful on this system.

% \subsection{ooc ---preprocessor for object-oriented coding in ANSI C}
\subsection{ooc:\prop{ANSI C} 面向对象编码预处理器}
\begin{lstlisting}[morekeywords={ooc}]
	ooc [option ...] [report ...] description target ...
\end{lstlisting}
\cemph{ooc} is an \cemph{awk} program which reads class descriptions and
performs the routine coding tasks necessary to do object-oriented coding in
\prop{ANSI C}. Code generated by \cemph{ooc} is controlled by reports which
may be changed. This manual page describes the effects of the standard
reports.

\cemph{description} is a class name. \cemph{ooc} loads a class description
file with the name\linebreak \cemph{description}.\ccbf{d} and recursively
class description files for all superclasses back to the root class. If
\ccbf{-h} or \ccbf{-r} is specified as a \cemph{target}, a C header file for
the public interface or the private representation of \cemph{description} is
witten to standard output. If \cemph{source}.\ccbf{dc} or \ccbf{-} is
specified as a \cemph{target}, \ccbf{\#include} statements for the
\cemph{description} header files are written to standard output. If
\ccbf{-dc} is specified as a \cemph{target}, a source skeleton for
\cemph{description} is written to standard output, which contains all
possible methods.


% \subsection{\cemph{Lexical Conventions}}
\subsection{\emph{词法}}
<++>

% \subsection{\cemph{Class Description File}}
\subsection{\emph{类描述文件}}
<++>

% \subsection{\cemph{Preprocessing}}
\subsection{\emph{预处理}}
<++>

% \subsection{\cemph{Tags}}
\subsection{\emph{标签}}
<++>

% \subsection{\cemph{Report File}}
\subsection{\emph{报告文件}}
<++>

% \subsection{\cemph{Environment}}
\subsection{\emph{环境}}
<++>

% \section{Functions}
\section{函数}
<++>

% \subsection{retrieve ---get object from file}
\subsection{retrieve:从文件获取对象}
<++>

% \section{Root Classes}
\section{根类}
<++>

% \subsection{intro ---introduction to the root classes}
\subsection{简介:根类入门}
<++>

% \subsection{Class Class: Object ---root metaclass}
\subsection{Class Class: Object ---root metaclass}
<++>

% \subsection{Class Exception: Object ---manage a stack of exception handlers}
\subsection{Class Exception: Object ---manage a stack of exception handlers}
<++>

% \subsection{Class Object ---root class}
\subsection{Class Object ---root class}
<++>

% \section{GUI Calculator Classes}
\section{GUI计算器类}
<++>

% \subsection{intro ---introduction to the calculator application}
\subsection{intro ---introduction to the calculator application}
<++>

% \subsection{lcClass Crt: lc ---input/output objects for curses}
\subsection{lcClass Crt: lc ---input/output objects for curses}
<++>

% \subsection{Class Event: Object ---input item}
\subsection{Class Event: Object ---input item}
<++>

% \subsection{lcClass: Class Ic: Object ---basic input/output/transput objects}
\subsection{lcClass: Class Ic: Object ---basic input/output/transput objects}
<++>

% \subsection{Class Xt: Object ---input/output objects for X11}
\subsection{Class Xt: Object ---input/output objects for X11}
<++>

A quick brown fox jumps over the lazy dogs.
A quick brown fox jumps over the lazy dogs.
A quick brown fox jumps over the lazy dogs.

A quick brown fox jumps over the lazy dogs.
A quick brown fox jumps over the lazy dogs.
A quick brown fox jumps over the lazy dogs.

% vim: set syntax=tex ts=4 sw=4 tw=76 fo+=Mm cc=+2 :

