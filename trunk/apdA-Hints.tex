% # Copyright (C) 2009-2011, suxpert, All right reserved.
% # -*- coding: utf-8 -*-
% !TEX encoding = UTF-8 Unicode

% Version Control System Information: Subversion, host on Google Code;
% FileID:		$Id$;
% FileDate:		$Date$;
% FileRevision:	$Revision$

% \chapter{\prop{ANSI-C} Programming Hints}
\chapter{\prop{ANSI-C} 编程提示}
\label{apd:CProgrammingHints}
C was originally defined by Dennis Ritchie in the appendix of \cite{KR78}.
The \prop{ANSI-C} standard \cite{ANSI} appeared about ten years later and
introduced certain changes and extensions. The differences are summarized
very concisely in appendix C of \cite{KR88}. Our style of object-oriented
programming with \prop{ANSI-C} relies on some of the extensions. As an aid
to classic C programmers, this appendix explains those innovations in
\prop{ANSI-C} which are important for this book. The appendix is certainly
not a definition of the \prop{ANSI-C} programming language.

% \section{Names and Scope}
\section{变量名和作用域}
\prop{ANSI-C} specifies that names can have almost arbitrary length. Names
starting with an underscore are reserved for libraries, i.e., they should
not be used in application programs.

Globally defined names can be hidden in a translation unit, i.e., in a
source file, by using \ccode{static}:
\begin{lstlisting}
static int f(int x) {...}	// Only visible in source file
int g;						// visible throughout the program
\end{lstlisting}
Array names are constant addresses which can be used to initialize pointers
even if an array references itself:
\begin{lstlisting}
struct table{ struct table * tp; }
	v[] = { v, v+1, v+2 };
\end{lstlisting}
It is not entirely clear how one would code a forward reference to an object
which is still to be hidden in a source file. The following appears to be
correct:
\begin{lstlisting}
extern struct x object;		// forward reference
f() { object = value; }		// using the reference
static struct x object;		// hidden definition
\end{lstlisting}

% \section{Functions}
\section{函数}
\prop{ANSI-C} permits ---but does not require ---that the declaration of a
function contains parameter declarations right inside the parameter list.
If this is done, the function is declared together with the type of its
parameters. Parameter names may be specified as part of the function
declaration, but this has no bearing on the parameter names used in the
function definition.
\begin{lstlisting}
double sqrt();				// classic version
double sqrt(double);		// ANSI-C
double sqrt(double x);		// ... with parameter names
int getpid(void);			// no parameters, ANSI-C
\end{lstlisting}
If an \prop{ANSI-C} function prototype has been introduced, an \prop{ANSI-C}
compiler will try to convert argument values into the types declared for the
parameters.

Function definitions may use both variants:
\begin{lstlisting}
double sqrt(double arg)		// ANSI-C
{ ... }
double sqrt(arg)			// classic
	double arg;
{ ... }
\end{lstlisting}
There are exact rules for the interplay between \prop{ANSI-C} and classic
prototypes and definitions; however, the rules are complicated and
error-prone. It is best to stick with \prop{ANSI-C} prototypes and
definitions, only.

With the option \ccbf{-Wall} the \prop{GNU-C} compiler warns about calls to
functions that have not been declared.

% \section{Generic Pointers ---\cemph{void *}}
\section{通用指针:\cemph{void *}}
Every pointer value can be assigned to a pointer variable with type
\ccode{void *} and vice versa, except for \ccode{const} qualifiers. The
assignments do not change the pointer value. Effectively, this turns off
type checking in the compiler:
\begin{lstlisting}
int iv[] = {1, 2, 3};
int * ip = iv;				// OK, same type
void * vp = ip;				// OK, arbitrary to void *
double * dp = vp;			// OK, void * to arbitrary
\end{lstlisting}
\ccbf{\%p} is used as a format specification for \ccbf{printf()} and
(theoretically) for \ccbf{scanf()} to write and read pointer values. The
corresponding argument type is \ccode{void *} and thus any pointer type:
\begin{lstlisting}
void * vp;
	printf("%p\n", vp);		// display value
	sanf("%p", & vp);		// read value
\end{lstlisting}
Arithmetic operations involving \ccode{void *} are not permitted:
\begin{lstlisting}
void * p, ** pp;
	p + 1					// wrong
	pp + 1					// OK, pointer to pointer
\end{lstlisting}
The following picture illustrates this situation:
% TODO: Draw the picture.

% \section{\cemph{const}}
\section{\cemph{const}}
\ccode{const} is a qualifier indicating that the compiler should not permit
an assignment. This is quite different from truly constant values.
Initialization is allowed; \ccode{const} local variables may even be
initialized with variable values:
\begin{lstlisting}
int x = 10;
int f(){ const int xsave = x; ... }
\end{lstlisting}
One can always use explicit typecast operations to circumvent the compiler
checks:
\begin{lstlisting}
const int cx = 10;
	(int) cx = 20;			// wrong
	* (int *) & cx = 20;	// not forbidden
\end{lstlisting}
These conversions are sometimes necessary when pointer values are assigned:
% TODO: another fucking figure
\begin{lstlisting}
const void * vp;

int * ip;
int * const p = ip;				// OK for local variable

	vp = ip;					// OK, blocks assignment
	ip = vp;					// wrong, allows assignment
	ip = (void *) vp;			// OK, brute force
	* (const int **) & ip = vp; // OK, overkill
	p = ip;						// wrong, pointer is blocked
	* p = 10;					// OK, target is not blocked
\end{lstlisting}
\ccode{const} normally binds to the left; however, \ccode{const} may be
specified before the type name in a declaration:
\begin{lstlisting}
int const v[10];			// ten constant elements
const int * const cp = v;	// constant pointer to constant value
\end{lstlisting}
\ccode{const} is used to indicate that one does not want to change a value
after initialization or from within a function:
\begin{lstlisting}
char * strcpy(char * target, const char * source);
\end{lstlisting}
The compiler may place global objects into a write-protected segment if they
have been completely protected with \ccode{const}. This means, for example,
that the components of a structure inherit \ccode{const}:
\begin{lstlisting}
const struct { int i; } c;
	c.i = 10;				// wrong
\end{lstlisting}
This precludes the dynamic initialization of the following pointer, too:
\begin{lstlisting}
void * const String;
\end{lstlisting}
It is not clear if a function can produce a \ccode{const} result.
\prop{ANSI-C} does not permit this. \prop{GNU-C} assumes that in this case
the function does not cause any side effects and only looks at its arguments
and neither at global variables nor at values behind pointers. Calls to this
kind of a function can be removed during common subexpression elimination in
the compilation process.

Because pointer values to \ccode{const} objects cannot be assigned to
unprotected pointers, \prop{ANSI-C} has a strange declaration for
\ccbf{bsearch()}:
\begin{lstlisting}
void * bsearch(const void * key
	const void * table, size_t nel, size_t width,
	int (* cmp) (const void * key, const void * elt));
\end{lstlisting}
\ccbf{table[]} is imported with \ccode{const}, i.e., its elements cannot be
modified and a constant table can be passed as an argument. However, the
result of \ccbf{bsearch()} points to a table element and does not protect it
from modification.

As a rule of thumb, the parameters of a function should be pointers to
\ccode{const} objects exactly if the objects will not be modified by way of
the pointers. The same applies to pointer variables The result of a function
should (almost) never involve \ccode{const}.

% \section{\cemph{typedef} and \cemph{const}}
\section{\cemph{typedef} 和 \cemph{const}}
\ccode{typedef} does not define macros. \ccode{const} may be bound in
unexpected ways in the context of a \ccode{typedef}:
\begin{lstlisting}
const struct Class { ... } * p;	// protects contents of structure
typedef struct Class { ... } * ClassP;
const ClassP cp;			// contents open, pointer protected
\end{lstlisting}
How one would protect and pass the elements of a matrix remains a puzzle:
\begin{lstlisting}
main()
{
	typedef int matrix [10][20];
	matrix a;
	int b [10][20];

	int f(const matrix);
	int g(const int [10][20]);

	f(a);
	f(b);
	g(a);
	g(b);
}
\end{lstlisting}
There are compilers that do not permit any of the calls...

% \section{Structures}
\section{结构体}
Structures collect components of different types. Structures, components,
and variables may all have the same name:
\begin{lstlisting}
struct u { int u; double v; } u;
struct v { double u; int v; } * vp;
\end{lstlisting}
Structures components are selected with a period for structure variables and
with an arrow for pointers to structures:
\begin{lstlisting}
u.u = vp -> v;
\end{lstlisting}

A pointer to a structure can be declared even if the structure itself has
not yet been introduced. A structure may be declared without objects being
declared:
\begin{lstlisting}
struct w * wp;
struct w { ... };
\end{lstlisting}
A structure may contain a structure:
\begin{lstlisting}
struct a { int x; };
struct b { ... struct a y; ... } b;
\end{lstlisting}
The complete sequence of component names is required for access:
\begin{lstlisting}
b.y.x = 10;
\end{lstlisting}
The first component of a structure starts right at the beginning of the
structure; therefore, structures can be lengthened or shortened:
\begin{lstlisting}
struct a { int x; };
struct c { sturct a a; ... } c, * cp = & c;
struct a * ap = & c.a;

	assert( (void *) ap == (void *) cp );
\end{lstlisting}
\prop{ANSI-C} permits neither implicit conversions of pointers to different
structures nor direct access to the components of an inner structure:
\begin{lstlisting}
ap = cp;					// wrong
c.x, cp -> x				// wrong
cp -> a.x					// OK, fully specified
( (sturct a *) cp ) -> x	// OK, explicit conversion
\end{lstlisting}

% \section{Pointers to Functions}
\section{函数指针}
The declaration of a pointer to a function is constructed from the
declaration of a function by adding one level of indirection, i.e., a
\ccbf{*} operator, to the function name. Parentheses are used to control
precedence:
\begin{lstlisting}
void * 		f	(void *);		// function
void * (* fp)	(void *) = f;	// pointer to function
\end{lstlisting}
These pointers are usually initialized with function names that have been
declared earlier. In function calls, function names and pointers are used
identically:
\begin{lstlisting}
int x;
	f		(& x);			// using a function name
	fp		(& x);			// using a pointer, ANSI-C
	(* fp)	(& x);			// using a pointer, classic
\end{lstlisting}
A pointer to a function can be a structure component:
\begin{lstlisting}
struct Class { ...
	void * (* ctor) (void * self, va_list * app);
... } * cp, ** cpp;
\end{lstlisting}
In a function call, \ccbf{->} has precedence over the function call, but is
has no precedence over dereferencing with \ccbf{*}, i.e., the parentheses
are necessary in the second example:
\begin{lstlisting}
cp -> ctor ( ... );
(* cpp) -> ctor ( ... );
\end{lstlisting}

% \section{Preprocessor}
\section{预处理器}
\prop{ANSI-C} no longer expands \ccode{#define} recursively; therefore,
function calls can be hidden or simplified by macros with the same name:
\begin{lstlisting}
#define malloc(type) (type *) malloc(sizeof(type))
int * [ = malloc(int);
\end{lstlisting}
If a macro is defined withe parameters, \prop{ANSI-C} only recognizes its
invocation if the macro name appears before a left parentheses; therefore,
macro recognition can be suppressed in a function header by surrounding the
function name with an extra set of parentheses:
\begin{lstlisting}
#include <stdio.h>				// defines putchar(ch) as a macro
int (putchar) (int ch) { ... }	// name is not replaced
\end{lstlisting}
Similarly, the definition of a parametrized macro no longer collides with a
variable of the same name:
\begin{lstlisting}
#define x(p) (((const struct Object *)(p)) -> x)
int x = 10;						// name is not replaced
\end{lstlisting}

% \section{Verification ---\cemph{assert.h}}
\section{断言:\cemph{assert.h}}
<++>

% \section{Global Jumps ---\cemph{setjmp.h}}
\section{全局跳转:\cemph{setjmp.h}}
<++>

% \section{Variable Argument Lists ---\cemph{stdarg.h}}
\section{变长参数列表:\cemph{stdarg.h}}
<++>

% \section{Data Types ---\cemph{stddef.h}}
\section{数据类型:\cemph{stddef.h}}
<++>

% \section{Memory Management ---\cemph{stdlib.h}}
\section{内存管理:\cemph{stdlib.h}}
<++>

% \section{Memory Functions ---\cemph{string.h}}
\section{内存函数:\cemph{string.h}}
<++>

A quick brown fox jumps over the lazy dogs.
A quick brown fox jumps over the lazy dogs.
A quick brown fox jumps over the lazy dogs.

A quick brown fox jumps over the lazy dogs.
A quick brown fox jumps over the lazy dogs.
A quick brown fox jumps over the lazy dogs.

% vim: set syntax=tex ts=4 sw=4 tw=76 fo+=Mm cc=+2 :

