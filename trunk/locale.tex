% # Copyright (C) 2009-2014 the Fandol Team, All wrongs reserved.
% # -*- coding: utf-8 -*-
% !TEX encoding = UTF-8 Unicode

% Version Control System Information: Subversion, host on Google Code;
% FileID:		$Id$;
% FileDate:		$Date$;
% FileRevision:	$Revision$

% In principle, this file can be redistributed and/or modified under
% the terms of the GNU Public License, version 3 or later.
%
% However, this file is supposed to be a template to be modified
% for your own needs. For this reason, if you use this file as a
% template and not specifically distribute it as part of a another
% package/program, I grant the extra permission to freely copy and
% modify this file as you see fit and even to delete this copyright
% notice.

% When use CJK, all settings which contains Chinese Characters must be
% within CJK environment. But \XeTeX{} or Lua\TeX{} does not need this.
% For better compatibility, all settings about Chinese Locale are separated
% in this file, and should included in the control file, e.g. paper.tex.

\renewcommand{\chaptername}{第\CJKnumber{\thechapter}章}
% \renewcommand{\chaptername}{第{\thechapter}章}
\newcommand{\sectionname}{节}
\renewcommand{\figurename}{图}
\renewcommand{\tablename}{表}
\renewcommand{\bibname}{参考文献}
\renewcommand{\contentsname}{目\quad 录}
\renewcommand{\listfigurename}{图~目~录}
\renewcommand{\listtablename}{表~目~录}
\renewcommand{\indexname}{索\quad 引}

\newcommand{\prechaptername}{第}
\newcommand{\postchaptername}{章}
\newcommand{\chapterformat}
{\prechaptername\CJKnumber{\thechapter}\postchaptername}

\titleformat{\chapter}[display]
{\normalfont\flushright\Huge\sffamily\bfseries}{\chapterformat}{20pt}{\Huge}
% \titlespacing{\chapter}{0pt}{-20pt}{25pt}

\titlecontents{chapter}[0em]{\addvspace{1.5ex}\bfseries}
{\prechaptername\CJKnumber{\thecontentslabel}\postchaptername\quad}
{\hspace*{-0.0em}}
{\hfill{}\contentspage}[\addvspace{0.5ex}]

% \titleformat{\chapter}[block]{\center\Huge}{\chaptername}{20pt}{}

% \renewcommand{\contentsname}{
% \begin{center}目\quad 录\end{center}
% }
\setcounter{tocdepth}{1} % 目录深度

% \renewcommand{\listfigurename}{图形清单}
% \renewcommand{\listtablename}{表格清单}
% \renewcommand{\figurename}{图}
% \renewcommand{\tablename}{表}
% \renewcommand{\thefigure}{~\arabic{section}.\arabic{subsection}.\arabic{figure}~}

% \renewcommand{\abstractname}{\large 摘\quad 要}

% \theoremstyle{definition}
% \newtheorem{dfn}{定义}[section]
% \theoremstyle{plain}
% I don't know which is the right one, the plain style which theorem looks
% like, but I think something like problem shouldn't use a Italy font
% \newtheorem{thm}{\bf 定理}[section]
% \newtheorem{lem}[theorem]{引理}
% \newtheorem{cor}[theorem]{推论}
% \newtheorem{pro}{问题}
% \newtheorem{prf}{\bf 证明:}
% \theoremstyle{example}
% \newtheorem{eg}{例}[section]
% \theoremstyle{remark}
% \newtheorem{rmk}{注记}

\renewcommand{\appendixname}{附录}
\renewcommand{\bibname}{参考文献}
% \renewcommand{\refname}{参考文献}

% \renewcommand{\chaptername}[2]{第~\thechapter~章}
% \titleformat{\section}{\centering\Large\bfseries}{第\,\thesection\,章}{1em}{}
% \usepackage{xCJKnumb} % If you use xeCJK, just pls use CJKnumber instead.
% \titleformat{\section}{\centering\Large\bfseries}
% {第\xCJKnumber{\thesection}章}{1em}{}

% \usepackage{titletoc}
% \titlecontents{section}[0pt]{\vspace{.5\baselineskip}\bfseries}
%     {第\xCJKnumber{\thecontentslabel}章\quad}{}
%     {\hspace{.5em}\titlerule*[10pt]{$\cdot$}\contentspage}

% vim: set syntax=tex ts=4 sw=4 tw=76 fo+=Mm cc=+2 noundofile nobackup :

